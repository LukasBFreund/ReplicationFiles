%---------------------------------------------------------------------------------------
%
%	Template: Articles
% 	Lukas Freund
%	Last updated: June 2019
%
%----------------------------------------------------------------------------------------

%---------------------------------------------------------------------------------------
% Main parameters
%----------------------------------------------------------------------------------------

\documentclass[12pt]{article}
%----------------------------------------------------------------------------------------
% Settings
%----------------------------------------------------------------------------------------
\usepackage[authoryear, round, longnamesfirst]{natbib}
\def\citeapos#1{\citeauthor{#1}'s (\citeyear{#1})}
\def\citeapost#1{\citeauthor{#1}'s}%

%----------------------------------------------------------------------------------------
% Packages
%----------------------------------------------------------------------------------------
\usepackage{setspace}
\usepackage{amsmath}
\usepackage{mathtools}
\usepackage{amsthm} 
\usepackage{amssymb}
\usepackage{mathptmx}
\usepackage{parskip}
\usepackage{tikz}
\usetikzlibrary{positioning}
\usepackage{color}
\usepackage{xcolor}
\usepackage{graphicx} 
\usepackage{booktabs} 
\usepackage{amssymb}
\usepackage{longtable,booktabs} 
\usepackage{marvosym}
\usepackage{csquotes}
\usepackage{pgfplots}
\usepackage{enumitem}
\usepackage{psfrag}
\usepackage{epstopdf}
\usepackage{subcaption}
\usepackage{ctable}
\usepackage[flushleft]{threeparttable}
\usepackage{microtype}
\usepackage{wrapfig}
\usepackage[toc,page]{appendix}

%----------------------------------------------------------------------------------------
% Extra commands
%----------------------------------------------------------------------------------------

\newcommand\highlight[1]{\textcolor{LFBlue}{\textbf{#1}}}
\newcommand\mathhighlight[1]{\textcolor{LFBlue}{\boldsymbol{#1}}}
\newcommand\draftnote[1]{\textcolor{LightGrey}{\textnormal{#1}}}
\newcommand\cites[1]{\citeauthor{#1}'s\ (\citeyear{#1})}

\DeclareMathOperator*{\maxi}{max}
\DeclareMathOperator*{\mini}{min}
\newcommand{\Cov}{\mathrm{Cov}}
\newcommand{\Var}{\mathrm{Var}}
\newcommand{\distributed}{\mathrel{\overset{\makebox[0pt]{\mbox{\normalfont\tiny\sffamily iid}}}{\thicksim}}}

\theoremstyle{plain}
\newtheorem{remark}{Remark}
\newtheorem{proposition}{Proposition}
\newtheorem{lemma}{Lemma}
\newtheorem{corollary}{Corollary}
\newtheorem{definition}{Definition}
\newtheorem{assumption}{Assumption}

%----------------------------------------------------------------------------------------
% Design
%----------------------------------------------------------------------------------------
	
% General specs
\usepackage[margin=1in]{geometry}
\onehalfspacing
\usepackage{url}
\setcounter{secnumdepth}{5}
\usepackage{color}   %May be necessary if you want to color links
\usepackage[bottom]{footmisc}
\usepackage{hyperref}
\hypersetup{
	colorlinks=true, % set true if you want colored links
	linktoc=all,     % set to all if you want both sections and subsections linked
	linkcolor=OxfordBlue2,  %choose some color if you want links to stand out
	citecolor = OxfordBlue2,
	urlcolor  = OxfordBlue2,
}
	
% Define some colors
	\definecolor{LFBlue}{RGB}{4,30,150} 
	\definecolor{OxfordBlue2}{RGB}{4,30,100} 
	\definecolor{LightGrey}{rgb}{0.3686, 0.5255, 0.6235} % (secondary)
	
% Plots
\pgfplotsset{yticklabel style={text width=3em,align=right}}

\newcommand\blfootnotea[1]{%
	\begingroup
	\renewcommand\thefootnote{$*$}\footnote{#1}%
	\addtocounter{footnote}{-1}%
	\endgroup
}
\newcommand\blfootnoteb[1]{%
	\begingroup
	\renewcommand\thefootnote{$\dag$}\footnote{#1}%
	\addtocounter{footnote}{-1}%
	\endgroup
}
\newcommand\blfootnotec[1]{%
	\begingroup
	\renewcommand\thefootnote{$\ddagger$}\footnote{#1}%
	\addtocounter{footnote}{-1}%
	\endgroup
}
%----------------------------------------------------------------------------------------
%	MAIN DOC
%----------------------------------------------------------------------------------------

\begin{document}
\doublespacing
%----------------------------------------------------------------------------------------
%	TITLE PAGE
%----------------------------------------------------------------------------------------
\thispagestyle{empty}
\vspace*{-1cm}
\begin{center}
	{\Large Replication files for:\\ ``Workers, Capitalists, and the Government: \\Fiscal Policy and Income (Re)Distribution''}
	 \\
%	\vspace*{2cm}
	\hspace*{-10cm}

		\centering{
		{\large Cristiano Cantore}\blfootnoteb{Bank of England, CfM \& University of Surrey. Email: \href{mailto:cristiano.cantore@gmail.com}{cristiano.cantore@gmail.com}. Web: \href{https://www.cristianocantore.com/}{cristianocantore.com}.}
	}
	%	\end{minipage}% This must go next to `\end{minipage}`
	\hspace*{1cm}
	%	\begin{minipage}{.5\textwidth}
	\centering{
		{\large Lukas B. Freund}\blfootnotec{University of Cambridge. Email: \href{mailto:lukas.beat.freund@gmail.com}{lukas.beat.freund@gmail.com}. Web: \href{https://www.lukasfreund.com/}{lukasfreund.com}.}
		
	}

\hspace*{-10cm}
			\vspace*{-0.5cm}

 October 2020
			\vspace*{1cm}

\end{center}


%----------------------------------------------------------------------------------
%	MAIN CONTENT
%----------------------------------------------------------------------------------
This document describes how to use the replication codes accompanying the article ``Workers, Capitalists, and the Government: Fiscal Policy and Income (Re)Distribution.'' Corresponding to sections 2 and 3 of the paper, there are two main folders that respectively consider the partial equilibrium consumption-savings problem and the proposed, new two-agent New Keynesian (TANK) model with capitalists and workers.

Should you have any questions or comments, please feel free to contact us under the indicated email addresses.

\section{Partial equilibrium iMPCs}
\subsection{Description of main files}
The files contained in this folder are as follows.

\path{main_PE_MPCs_Analytical_Plotting.m} \\
This file computes and plots iMPCs for a one-off income shock in a partial equilibrium consumption-savings model with portfolio adjustment costs, using the analytical solution (accurate to first-order) provided in the main text. The user can choose the anticipation horizon $s$. The structural parameters are calibrated to match the empirical evidence, as described in the main text.

\path{main_PE_MPCs_Analytical_Matching.m}\\
This file implements computes the values of population share $\lambda$ and portfolio adjustment cost parameter $\psi$ to match targets from micro consumption data.

\path{main_PE_MPCs_LTI.m}\\
Complementing the analytical solution, this file solves the partial equilibrium model numerically. We use the linear time iteration approach of \citet{rendahlLinearTimeIteration2017a}. We also used this approach to compute the iMPCs in the model with habits (see parameter `sPar.hab').

\path{main_PE_Rates_Analytical_Plotting.m}\\
This file is similar to \path{main_PE_MPCs_Analytical_Plotting.m}, the only difference being that it looks at interest rate shocks.

\path{main_PE_Rates_Analytical_Extra.m}\\
This file plots some additional figures, such as the interest rate elasticity of consumption as a function of $\psi$.

\path{sSettings.mat}\\
This structure object stores generic, design-related settings to ensure consistency across figures.

\subsection{Exact replication of figures in the paper}
\textbf{Figure 2a.} Run \path{main_PE_MPCs_Analytical_Plotting.m} with \path{s=0} and \path{optionModel} set to \path{H}. \\
\textbf{Figure 2b.} Run \path{main_PE_MPCs_Analytical_Plotting.m} with \path{s=0} and \path{optionModel} set to \path{W}. \\
\textbf{Figure 3a.} Run \path{main_PE_MPCs_Analytical_Plotting.m} with \path{s=3} and \path{optionModel} set to \path{H}. \\
\textbf{Figure 3b.} Run \path{main_PE_MPCs_Analytical_Plotting.m} with \path{s=3} and \path{optionModel} set to \path{W}. 

\section{TANK models}
\subsection{Description of main files}
The sub-folder \path{Dynare master codes} contains Dynare implementations \citep{adjemianDynareReferenceManual2018} of the different TANK models.  The remaining sub-folders contain files to reproduce the exact figures shown in the main text. Regarding the former, the files contained in this sub-folder are as follows.

\bigskip \noindent  \underline{Baseline models.}\\
\path{tank_uh.mod} \\
Dynare code to replicate the baseline TANK-UH model linearized around 0 steady-state government spending/debt.

 \path{tank_uw.mod} \\
Dynare code to replicate the baseline TANK-UW model linearized around 0 steady-state government spending/debt.

\path{tank_cw.mod} \\
Dynare code to replicate the baseline TANK-CW model linearized around 0 steady-state government spending/debt.

\bigskip \noindent  \underline{Medium-scale models.}\\
\path{tank_cw_ms.mod} \\
Dynare code to replicate the medium scale version of the TANK models used in the paper. The file includes a pre-processor macro variable at the top that can be set to choose between different version of the model, i.e., RANK, TANK-UH, TANK-UW, TANK-CH and TANK-CW.


\subsection{Exact replication of figures in the paper}
\textbf{Figure 5.} Run \path{~TANK models/fig5/run_fig5.m}.  \\
\textbf{Figure 6a.} Run \path{~TANK models/fig6/figure 6a/run_fig6a.m}.  \\
\textbf{Figure 6b.} Run \path{~TANK models/fig6/figure 6a/run_fig6a.m}.  \\
\textbf{Figure 7.} Run \path{~TANK models/fig7/run_fig7.m}.  \\
\textbf{Table 3.} Run \path{~TANK models/table3/simple models/run_table3.m} for the first four columns and \path{~TANK models/table3/medium scale/run_table3MS.m} for the last four columns.  

\section{Additional materials}
\subsection{Empirical evidence}
\textbf{Figure 1.} Figure 1a relies on data from \citet{fagerengMPCHeterogeneityHousehold2018}, analyzed in and kindly shared by \citet{auclertIntertemporalKeynesianCross2018}. Figure 1b replicates Figure 1 in \citet{jappelliFiscalPolicyMPC2014} using data distributed through \href{http://doi.org/10.3886/E116432V1}{openICPSR}. We therefore do not include these files in this replication kit.

\textbf{Figure 4.} Data and code for this figure are in the sub-folder \path{VAR}. To reproduce the figure, run  \path{VAR_Plotting.m}. That same file also permits reproducing the VAR-based figures shown in the online appendix.

\subsection{Online appendix}
This folder contains various sub-folders with codes that replicate the figures in the online appendix.
%----------------------------------------------------------------------------------
\renewcommand\refname{References for documentation}
\bibliography{references}           
\bibliographystyle{natbib}     

\end{document}


